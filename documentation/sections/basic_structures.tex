\section{Basic structures}
\label{Sec:basic_structures}

The project started implementing all the basic structures needed by the algorithm. All the structures are as generic as possible to encourage reuse and are dynamically allocated and are implemented using lists to have no static limitations. \\
Each structure have a create fucntion for allocation and a destroy one for disallocation of the memory.

\subsection{hash table}

Is implemented with multiplicative modular method with golden ratio:

\begin{equation}
    golden\_ratio = (\sqrt{5} - 1)/2
\end{equation}
\begin{equation}
    P = 8191
\end{equation}
\begin{equation}
    hash(key, module) = ((key*golden\_ratio)\%P)\%module
\end{equation}

The structures re-allocates itself when reaches the 3/4 of the maximum capacity, in case of collisions concatenate the elements in a list (keeps in the first place the last inserted item).

\subsection{heap}

The heap structures uses hash table for faster addressing and is used as a priority queue with the possibility to have MAX or MIN priority as first element

\subsection{queue}

A generic list with Head and Tail pointers with a generic item. FIFO queue with head extraction and tail insertion

\subsection{stack}

LIFO implementation with both insertion and extraction on the head

\subsection{graph}

Generic Graph with possibility to be UNDIRECTED or DIRECTED, have a pointer to generic data (possibility to manipulate passing custom funcitons eg. 2d data). \\
Each vertex has a true\_cost (the cost of the path from start to that vertex) and a heuristic\_cost (exstimation of the cost from that node to the destination node).
The graph also contains a Hash table of all the nodes used to efficiently find a node (a good improvement in the graph generation).