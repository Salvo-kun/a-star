\section{Graph reading}
\label{Sec:reading}

Starting from a .map file (standard format for maps representation, using ascii characters to represent walls and passable terrain)
the \verb|graph_generator.py| python script is able to create a .txt with the format required by our A* implementation. \\
In the benchmark section we have:
\begin{itemize}
	\item huge-graphs: graphs with millions of nodes, used for comparisons on large dimensions
	\item maze: graph with a small number of solutions (few path available from a point to another)
	\item street-maps: true city grid maps with different sizes
	\item small-graphs: generated by us used for testing
\end{itemize}
The weight of the edges can be randomized by the script in order to raise the level of difficulty in finding a path (all the grids have only edges with unitary weight).
\\
The file is processed sequentially, creating all the nodes first and then the edges.
