\section{Reading}
\label{Sec:reading}

Starting from a .map file (standard format for maps rappresentation, using ascii characters to represent walls and passable terrain)
the \verb|graph_generator.py| python script is able to create a .txt with the format required by our A-Star implementation. \\
In the benchmark section we have:
\begin{itemize}
	\item huge-graphs: graphs with millions of nodes, used for comparisions on large dimensions
	\item maze: graph with a small number of solutions (few path available from a point to another)
	\item street-maps: true city grid maps with different size
	\item small-graphs: generated by us used for testing
\end{itemize}

The weight of the edges can be randomized by the script in order to raise a level of difficulty in finiding a path (all the grids benchmark have only edge with weight 1)

% You would probably use a table or two. See Table \ref{table:example_table}
%     \begin{table}[t] % The t here tells it to align jump to the top. you can change this to your desired behaviour
% 	\centering	
% 		\tabcolsep = 0.01\textwidth
% 		\begin{tabular}{| m{0.1\textwidth} | M{0.15\textwidth} | M{0.15\textwidth}|} %Capital M is a custom definition. I used two here since usually you want the methods aligned to the left and the scores centered.
% 			\hline
% 			\centering\textbf{Method} & \textbf{Criterion 1} & \textbf{Criterion 2}
%  			\tabularnewline
%  			\hline
%  			Method 1 & score 1 & score 2 \\
%  			Method 2 & score 1 & score 2 \\
%  			Ours  & The best & The best \\
% 		    \hline
% 		\end{tabular}
% 	\caption{Example of a table }
% 	\label{table:example_table}
% \end{table}